% This file is part of the Radio Array Model project.
% Copyright 2012 David W. Hogg

\documentclass[12pt]{article}

\newcommand{\project}[1]{\textsl{#1}}
\newcommand{\CLEAN}{\project{CLEAN}}
\newcommand{\NAELC}{\project{NAELC}}

\newcommand{\set}[1]{\left\{{#1}\right\}}
\newcommand{\given}{\,|\,}
\newcommand{\expectation}[1]{\tilde{#1}}
\newcommand{\dd}{\mathrm{d}}

\begin{document}

...Interferometry data have these properties...positive reasons to do
what we are doing...

...\CLEAN\ is very wrong...negative reasons to do what we are doing...

We get from the telescope \emph{visibilities}, which for our purposes
are $N$ samples $d_n$ of the form
\begin{eqnarray}
d_n &\equiv& \set{a_n, b_n, t_n, u_n, v_n, w_n, Z_n}
\\
Z_n &\equiv& A_n + i\,B_n
\\
Z_n &\equiv& \left|Z_n\right|\,\exp(i\,\phi_n)
\quad,
\end{eqnarray}
where $a_n$ and $b_n$ are the identities of the two antennae used to
make the baseline used for sample $n$, $t_n$ is the time at which the
sample was taken, $(u_n, v_n)$ is the $u$--$v$-plane location of the
sample, $w_n$ can be ignored (for our purposes), and $Z_n$ is the
measured complex amplitude or \emph{visibility}.  In what follows, we
will assume that every aspect of the data is correctly calibrated.
Under this assumption, the $Z_n$ are really the data, and the $(u_n,
v_n)$ can be thought of as prior information.

Each sample comes from a baseline connecting two antennae; each
antenna has a system temperature, and there is also a correlator
temperature of some kind, so we expect a noise variance $\sigma_n^2$
on the complex $Z_n$
\begin{eqnarray}
\sigma_n^2 &\propto& T_a + T_b + T_{ab}
\quad ,
\end{eqnarray}
where $T_a$ is the noise temperature of antenna $a_n$, $T_b$ is the
noise temperature of antenna $b_n$, and $T_{ab}$ is an additional
correlator noise for this baseline.

The generative model for our purposes is as follows: There is a
time-independent intensity field $I(x,y\given\nu)$ as a function of
electromagnetic frequency $\nu$, defined in sky coordinates $(x,y)$ or
the $x$--$y$ plane.  At each $(u_n, v_n)$ sample location in the
$u$--$v$ plane the expected complex visibility is
\begin{eqnarray}
\expectation{Z_n} &=& \int I(x,y\given\nu)\,\exp(-2\pi\,i\,[u_n\,x + v_n\,y])\,\dd x\,\dd y
\end{eqnarray}
(HOGG: work out pre-factors like $\sqrt{\pi}$ etc.).

\begin{eqnarray}
-2\,\ln p(Z_n|I) &=& \frac{Z_n - \expectation{Z_n}}{\sigma_n^2}
\end{eqnarray}
ignoring an offset.

\end{document}
